\documentclass[12pt]{article}

%设置页边距
\usepackage{geometry}
\geometry{left=2.5cm,right=2.5cm,top=2.5cm,bottom=2.5cm}

%地址链接支持
\usepackage{hyperref}

%支持分栏
\usepackage{multicol}

%页眉页脚
\usepackage{fancyhdr}
\pagestyle{fancy}
\lhead{MSTvisualize Design Documentation}
\rhead{徐磊 2013011344}

%设置字体
\usepackage[no-math]{fontspec}
\usepackage{xunicode}
\usepackage{xltxtra}
\setmainfont[AutoFakeSlant]{Hiragino Sans GB W3}

%设置换行缩进
\usepackage[indentfirst=false,slantfont,boldfont]{xeCJK}
\usepackage{indentfirst}
\setlength{\parindent}{0.9cm}

%行距
\linespread{1.2}

\begin{document}

\title{MSTvisualize 设计文档}

\author{徐磊\thanks{\href{mailto:leopard.lie@gmail.com}{leopard.lie@gmail.com}}\quad 2013011344}

\date{2014年9月}
\maketitle

\section{功能介绍}
MSTvisualize应用程序的主要功能为求解在二维平面上点集的欧几里得最小生成树问题,并通过可视化的方式显示出来。支持二维平面点集的Delaunay三角剖分和Voronoi图的可视化显示。并且可以同时显示这些图形。

\section{设计分析}
\subsection{算法部分}
算法包括根据点集求解最小生成树、Delaunay三角剖分和Voronoi图。程序使用了符合GPL/LGPL开源协议的CGAL(Computational Geometry Algorithms Library)库,该库提供了二维平面的Delaunay三角剖分算法,并可以根据三角剖分求解Voronoi图。该库同时可以提供三维平面的相关算法,所以对库的使用使得未来对于支持三维空间的开发变得简单。基于三角剖分的最小生成树采用了kruskal算法。

算法部分通过Spantree类的封装,对外提供一下功能:
\begin{itemize}
\item 清空点集
\item 插入点集
\item 删除指定点集
\item 获取点集
\item 获取Delaunay三角剖分
\item 获取Voronoi图
\item 获取最小生成树
\end{itemize}

\subsection{交互部分}
View2D基于QWidget,作为二维情况下的界面和控制器。界面部分包括了大量重要的UI部件,包括左边绘图区域(Canvas)、右侧的选框(QCheckBox)、点列表(QTableView)以及工具栏(QToolBar)、菜单栏(QMenuBar)等。作为控制器,View2D包含了一个Spantree,存储了当前点集的状态。View2D负责接收来自UI部件的各种消息,对Spantree进行相关操作后对Canvas和CanvasPainter发出绘图相关的命令。

Canvas继承自QGraphicsView,内部包含一个绘图器(CanvasPainter),一个小地图(Guide),Canvas对方大缩小功能提供了封装。Canvas还会根据CanvasPainter的变化自动对Guide进行调整。

CanvasPainter继承自QGraphicsScene,保存了目前的图像,并提供了如下封装:
\begin{itemize}
\item 绘制点集
\item 绘制边集
\item 显示部件
\item 隐藏部件
\end{itemize}
CanvasPainter还接受双击事件,向View2D发送加点的消息。

Guide类继承自QGraphicsView,用作小地图的显示。接受鼠标移动事件,向Canvas发出重新定位CanvasPainter的消息。

QTableView类显示了目前的点集,在表格中选择点时,会向View2D发送高亮的消息。选中的点集可以按下erase按钮删除。

\section{效率提升}
为了提升程序效率,使用了多线程的技术,将算法部分加入一个独立的线程。对点集修改后,会自动计算对应的Delaunay三角剖分、Voronoi图和最小生成树,并将图形记录下了以备显示。一般情况下,Voronoi图在点数超过10000时较为缓慢,在其未计算完成时右侧对应的复选框显示为禁用状态,此时可以显示或隐藏其他的图形,或者加点、删除点、载入文件。多线程的使用避免了UI界面假死的现象,用户可以正常的操作界面。

抗锯齿功能在点数少于1000时自动开启,以获取流畅和优美的一个平衡。

尽管采用了大量措施提升效率,用户仍需要避免大数据下的频繁操作。

\section{说明}
\noindent开发框架:
\begin{itemize}
\item QT5(GPL/LGPL Software License)
\end{itemize}
库:
\begin{itemize}
\item boost(Boost Software License)
\item CGAL(GPL/LGPL Software License)
\end{itemize}
资源:
\begin{itemize}
\item 图标(\href{http://www.iconshock.com/}{http://www.iconshock.com/}, Free for commercial use)
\end{itemize}
\end{document}
编译:
\begin{itemize}
\item 请确保boost、CGAL及其所依赖的库已安装
\item 修改.pro文件中关于include和lib的相关内容
\item 执行qmake
\item 执行make
二进制版本适用于OS X 10.7及以上系统。
\end{itemize}